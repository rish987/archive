\begin{part}{Notation}
  To answer this question, let's establish some notation.

  \reflnenv{definition}{states}{Define}{ \statesdef}.

  \reflnenv{definition}{actions}{Define}{ \actionsdef}.
  
  \reflnenv{definition}{rewards}{Define}{ \rewardsdefg{\refln{note}{fnnot}{$\rewardsdefnot$} $\defeq$}{.}}

  \reflnenv{definition}{steplim}{Define}{ \steplimdef{.}}%

  \reflnenv{definition}{strs}{Define}{ \strsdef{.}}%

  \reflnenv{proof}{npol}{Note that}{ \npolprop{.}}%

  \reflnenv{definition}{inddef}{Define}{ \inddefdefg{\refln{note}{paramprob}{$\inddefdefp$} $\defeq$}{.}}%

  \reflnenv{definition}{pactdef}{Define}{ \pactdefdef{.}}%

  \reflnenv{definition}{trddef}{Define}{ \trddefdef{.}}%

  \reflnenv{definition}{rvst}{Define}{ \rvstdef{.}}%

  \reflnenv{definition}{rvat}{Define}{ \rvatdef{.}}%

  \reflnenv{definition}{psp}{Let}{ \pspdef{.}}%

  \reflnenv{definition}{pstr}{Let}{ \pstrdef{.}}%

  Because we follow a fixed \str,
  \reflnenv{definition}{pact}{we know that}{, \pactdeffull{,}}%
  \reflnenv{proof}{actcind}{from which it follows}{ that \actcindpropg{\text{$\actcindeq$}}{.}}%
  \reflnenv{definition}{ind}{We know that}{ \inddef{,}}%
  and
  \reflnenv{definition}{trd}{we know that}{ \trddef{,}}%
  \reflnenv{definition}{trdcind}{where}{ \trdcinddef{.}}
\end{part}
