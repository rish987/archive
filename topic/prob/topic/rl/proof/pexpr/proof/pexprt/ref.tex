\begin{proposition}
  \nrp 1
  We have that \pexprtprop{.}%
\end{proposition}

\begin{proof}
  \hrule
  \nrp 2
  {\it Base case.}
  When $\TT = 1$,
  we have \reflng{\rltpath/definition/ind}{by definition} that
  $\bca = \bcind$.

  \serule
  \nrp 3
  {\it Inductive hypothesis.}
  Assume that for $\TT - 1$ such that $1 \le \TT - 1 \le \T - 1$,
  the proposition holds. That is,
  $$ \pexprtihp \rell = ih \pexprtih.$$%
  \srule
  \nrp 4
  {\it Inductive step.}
  We want to show that the proposition holds for $\TT.$

  \nrp 5
  There are two cases to consider:
  \begin{itemize}
    \item \nrp 6 Case 1: $\pexprtihp \rell = co 0$. 

      \nrp 7
      Then,
      $$\pexprtp  \rellrf{=}{coz}{(co)->\mrefmargzimpl} 0,$$
      \nrp 8
      and
      \begin{align*}
        &\pexprt \\
        =\ &\pexprtih \times \\
        &\isact \times \istrd\\
        \relr = ih \ & \pexprtihp \times \isact \times \istrd \\
        \relr = co \ & 0 \relr = coz \pexprtp.
      \end{align*}
    \item \nrp 9 Case 2: $\pexprtihp \rell > ct 0$. 

      \nrp 10
      Then, 
      \ctext{$\isbf \rellrf{\ne}{daf}{(ct)->\mrefconddef} \undef$, $\isb \rellrf{\ne}{da}{(daf)->\mrefconddefimpl} \undef $,}
      \nrp 11
      and we have
      $$\isbf \rellr = cia (daf)->\mreflng{\rltpath/proof/actcind} \isb 
      \rellr = ad (da)->\mreflng{\rltpath/definition/pact} \isact.$$

      \nrp 12
      \reflng{\rltpath/definition/pactdef}{There are two cases} to consider:
      \begin{itemize}
        \item \nrp 13 Case 2.1: $\isact \rell = cto 1$.

          \nrp 14
          Then,
          \begin{align*}
            \ismid &\rellr = md {(ct)->\mrefcprtompr} \isbf \times \pexprtihp\\
                   &\relr = cia-ad \isact \times \pexprtihp\\
                   &\relr = cto \pexprtihp\\
                   &\rellr > mg ct 0.
          \end{align*}
          \nrp 15
          Therefore, 
          $$\iscf \rellrf{\ne}{dsf}{(md-mg)->\mrefconddef} \undef,\isc \rellrf{\ne}{ds}{(dsf)->\mrefconddefimpl} \undef$$
          \nrp 16
          and we have
          $$\iscf 
          \rellr = cis (dsf)->\mreflng{\rltpath/definition/trdcind} \isc 
          \rellr = sd (ds)->\mreflng{\rltpath/definition/trd} \istrd.$$
          \nrp 17
          Therefore,
          \begin{align*}
            \pexprtp 
            \relr = {(md-mg)->\mrefcprtompr,(ct)->\mrefcprtompr} \ & \pexprtihp \times \isbf \times\\& \iscf\\
            \relr = cia,cis \ & \pexprtihp \times \isb \times \isc\\
            \relr = ih,ad,sd \ & \pexprg{\TT}{\TT'}.
          \end{align*}
        \item \nrp 18 Case 2.2: $\isact \rell = ctt 0$.

          \nrp 19
          Then,
          \begin{align*}
            \ismid &\rellr = afd {(ct)->\mrefcprtompr} \isbf \times \pexprtihp\\
                   &\relr = cia-ad \isact \times \pexprtihp\\
                   &\rellr = afz ctt 0.
          \end{align*}
          \nrp 20
          Therefore,
          $$\pexprtp \rellr = cttz {(afd-afz)->\mrefmargzimpl} 0,$$
          \nrp 21
          so
          $$\pexprtp
          \relr = cttz \ 0 \ 
          \relr = ctt \pexprt.$$
      \end{itemize}
  \end{itemize}
  \hrule
\end{proof}
