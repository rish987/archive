\begin{note}
  \def\paramprob{probability parameterization}
  We use the notation $\Pr{\cdot}$ and $\PP(\cdot)$ to 
  distinguish between a probability and a ``\paramprob'',
  respectively.

  A \paramprob\ is simply a function mapping 
  to probability values, and is defined for every possible input.
  The form of the inputs are implicit in its definition.

  A probability, however, may or may not be defined, specifically
  when it is a conditional probability.

  When we correspond a probability with a \paramprob,
  we mean to say that, for a particular input, 
  they are equal when the probability is defined.

  For example, suppose $\SSPT{\SPT}{(\XV,\X),(\YV,\Y)}$ with any $\DD$.

  For all $\pinw{\XX, \YY \in \X\times\Y}$,
  if we correspond $\CPr{\XV = \XX}{\YV = \YY}$ with $\PP(\XX \mid \YY)$,
  we mean to say that when $\CPr{\XV = \XX}{\YV = \YY}$ is defined (that is, when $\MPr{\YV = \YY} > 0$),
  $$\CPr{\XV = \XX}{\YV = \YY} = \PP(\XX \mid \YY).$$%
\end{note}
