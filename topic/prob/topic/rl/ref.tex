\begin{center}
\nrp 1
{\LARGE \hlc[___gray]{Reinforcement Learning Theory}}\\
\vspace{10pt}
\normalsize 
Adapted from
{\it Neuro-Dynamic  Programming}
by \\
Dmitri P. Bertsekas
and
John N. Tsitsiklis.
\end{center}

\begin{part} {Physical intuition}
  \nrp 2
  Consider the following ``\brd'':\\
  \begin{center}
    \begin{tikzpicture}[board]
      \board{
        \tilecoords{0}{0}\actions\agent{}
      }
    \end{tikzpicture},
  \end{center}
  where there are a total of 9 ``\til{}s'' (\tikz[board, scale=0.1]{\tilecoords{0}{0}\tile{0}\tileidx{$\cdot$}}), 
  an ``\agt'' (\tikz[board, scale=0.7]{\tilecoords{0}{0}\agent{}}), and
  a set of ``\act{}s'' ($\rightarrow$) that the \agt\ can take at any \til.

  \nrp 3
  At every ``\stp'' in time, the agent is in one of the \til{}s,
  and must take one of the \act{}s. 

  \nrp 4
  At any \stp, if the \agt\ takes an \act\ in a \til, 
  it will be in a \til\ at the next \stp\ according to a ``\trd''
  (that only depends on the current \til\ and \act, not the current \stp).

  For example, the \agt\ could
  \begin{itemize}
    \def\tilesep{3.2}
    \def\tscale{0.6}
    \item take \act\ $\uparrow$ at \til\ $7$:
      \begin{center}
        \begin{tikzpicture}[board, scale=\tscale]
          \tilecoords{0}{1}
          \tileandidx{7}
          \tileup{draw=blue}
          \agent{}

          \tilecoords{0}{0}
          \tileandidx{4}

          \tilecoords{1}{1}
          \tileandidx{8}

          \tiletransition{7}{8}{below}{0.25}
          \tiletransition{7}{4}{left}{0.75}
        \end{tikzpicture},
      \end{center}
      and
    \item take \act\ $\rightarrow$ at \til\ $5$:
      \begin{center}
        \begin{tikzpicture}[board, scale=\tscale]
          \tilecoords{1}{1}
          \tileandidx{5}
          \tileright{draw=blue}
          \agent{}

          \tilecoords{0}{1}
          \tileandidx{4}

          \tilecoords{2}{1}
          \tileandidx{6}

          \tilecoords{1}{0}
          \tileandidx{2}

          \tilecoords{1}{2}
          \tileandidx{8}

          \tiletransition{5}{6}{below}{0.75}
          \tiletransition{5}{4}{above}{0.05}
          \tiletransition{5}{2}{left}{0.10}
          \tiletransition{5}{8}{right}{0.10}
        \end{tikzpicture},
      \end{center}
  \end{itemize}
  showing the \cprb\ of 
  being in another \til\ at 
  the next \stp\ according to the \trd.

  \nrp 5
  However, this begs the question, 
  what \til\ is the \agt\ at initially (\stp\ 1)? 
  To decide this, there is also an \ind\ over the \til{}s.

  \newpage
  \nrp 6
  Now, let's make things a bit more interesting. 

  Let every \til\ also have a \rwd\ that 
  the \agt\ recieves from it, 
  leaving our \brd\ looking something like:

  \begin{center}
    \begin{tikzpicture}[board]
      \fullboard{
        \tilecoords{0}{0}\actions\agent{}
      }
    \end{tikzpicture},
  \end{center}
  with the \rwd{}s 
  (\tikz[board, scale=0.5]{\tilecoords{0}{0}\tilerwdraw{$\cdot$}{(0,0)}}) shown.

  \nrp 7
  Also, let there be some fixed ``\lmt'' such that,
  when the current \stp\ reaches it, everything ends.

  \nrp 8
  Now, suppose you can ``program'' the \agt\ as follows: 
  for every \til\ and \stp, you can fix the \act\ that the \agt\ takes.
  For example, here's what you might program the \agt\ to do 
  at every \til\ for two different \stp{}s:
  \begin{center}
    \begin{tikzpicture}[board]
      \fullboardact{1}{} 3
    \end{tikzpicture}
    \hspace{1cm}
    \begin{tikzpicture}[board]
      \fullboardact{2}{.}
    \end{tikzpicture}.
  \end{center}
  We'll call each such programming a ``\str''. 

  \nrp 9
  We know that the \agt\ starts in some \til\ according to the \ind.
  After this, we follow the \str, 
  transitioning \til{}s according to the \trd, recieving \rwd{}s 
  at every \til\ until
  we reach the \lmt. Let's refer to this process as a ``\run'', and
  the sum of the \rwd{}s over all a \run's \til{}s
  as the ``\trwd''.

  \nrp 10
  Because there is some \ind\ and \trd\ 
  that are fixed over \run{}s, 
  we know there must be 
  some \atrwd\ for every \str.

  \nrp 11
  Now, consider the following question: 
  what \str\ has the greatest \atrwd?
\end{part}

\begin{part}{Notation}
  \nrp 12
  To answer this question, let's establish some notation.

  \nrp 13
  \reflnenv{definition}{states}{Define}{ \statesdef}.

  \nrp 14
  \reflnenv{definition}{actions}{Define}{ \actionsdef}.
  
  \nrp 15
  \reflnenv{definition}{rewards}{Define}{ \rewardsdefg{$\rewardsdefnot$ $\defeq$}{.}}

  \nrp 16
  \reflnenv{definition}{steplim}{Define}{ \steplimdef{.}}%

  \nrp 17
  \reflnenv{definition}{strs}{Define}{ \strsdef{.}}%

  \nrp 18
  \reflnenv{proof}{npol}{Note that}{ \npolprop{.}}%

  \nrp 19
  \reflnenv{definition}{inddef}{Define}{ \inddefdef{.}}%

  \nrp 20
  \reflnenv{definition}{pactdef}{Define}{ \pactdefdef{.}}%

  \nrp 21
  \reflnenv{definition}{trddef}{Define}{ \trddefdef{.}}%

  \nrp 22
  \reflnenv{definition}{rvst}{Define}{ \rvstdef{.}}%

  \nrp 23
  \reflnenv{definition}{rvat}{Define}{ \rvatdef{.}}%

  \nrp 24
  \reflnenv{definition}{psp}{Let}{ \pspdef{.}}%

  \nrp 25
  \reflnenv{definition}{pact}{We know that}{, \pactdeffull{,}}%
  \nrp 26
  \reflnenv{proof}{actcind}{from which it follows}{ that \actcindprop.}%

  \nrp 27
  \reflnenv{definition}{ind}{We know that}{ \inddef{,}}%

  \nrp 28
  \reflnenv{definition}{trd}{We know that}{ \trddef{,}}%
  \nrp 29
  \reflnenv{definition}{trdcind}{where}{ \trdcinddef{.}}
\end{part}

\begin{part}{Propositions}
  \nrp 30
  Let's now describe some preliminary results using this notation.

  \nrp 32
  \reflnenv{proof}{pexpr}{It can be shown}{ that, \pexprprop{.}}
\end{part}
