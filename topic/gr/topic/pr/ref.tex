\begin{center}
\nrp 1
{\LARGE \hlc[___gray]{Probability Theory}}\\
\end{center}
\begin{part} {Physical intuition}
  \def\AV{{\rm A}}
  \def\BV{{\rm B}}
  \def\R{\colorbox{red}{\rm R}}
  \def\B{\colorbox{blue}{\rm B}}
  \def\1{\colorbox{__gray}{1}}
  \def\0{\colorbox{__gray}{0}}
  \nrp 2
  Suppose that we have a set of ``\var{}s,''
  each with their own finite ``\dmn{}s'',
  which we'll collectively refer to as the ``\spt'':
  \def\tvar#1#2#3#4{
    \node[draw=black,inner sep=8mm,rectangle] (#1) at #2 {};
    \node[anchor=north west] at (#1.north west) {$#3$};
    \node[fill=___gray,minimum size=6mm,rectangle] at #2 {#4};
  }
  \begin{center}
    \begin{tikzpicture}

      \tvar{AR}{(0,0)}{\AV}{}
      \node[anchor=south] at (AR.north) {\var};

      \node[draw=black,minimum height=10mm,minimum width=15mm,rectangle] (DA) at (2,0) {};
      \node at ($(DA.west)!0.5!(DA.center)$) {\R};
      \node at ($(DA.east)!0.5!(DA.center)$) {\B};
      \node[anchor=south] at (DA.north) {\dmn};

      \tvar{BR}{(0,-2.5)}{\BV}{}
      \node[anchor=south] at (BR.north) {\var};

      \node[anchor=east] at ($(AR.west)!0.5!(BR.west) + (-0.6, 0)$) {\spt:};

      \node[draw=black,minimum height=10mm,minimum width=15mm,rectangle] (BD) at (2,-2.5) {};
      \node at ($(BD.west)!0.5!(BD.center)$) {\1};
      \node at ($(BD.east)!0.5!(BD.center)$) {\0};
      \node[anchor=south] at (BD.north) {\dmn};
    \end{tikzpicture},
  \end{center}
  \nrp 3
  where a ``\masn'' consists of associating some (but not all) of the \var{}s of the \spt\ 
  with something from their \dmn{}s:
  \begin{center}
    \begin{tikzpicture}
      \tvar{AR}{(0,0)}{\AV}{\B}
      \node[anchor=south] at (AR.north) {\masn};
    \end{tikzpicture},
  \end{center}
  \nrp 4
  a ``\jasn'' (``\asn'' for short) consists of associating each \var\ of the \spt\ with something from its \dmn:
  \def\tjasn#1#2#3#4#5{
    \tvar{#1}{#5}   {\AV}{#3}
    \tvar{#2}{($#5 + (0,-1.6)$)}{\BV}{#4}
  }
  \begin{center}
    \begin{tikzpicture}
      \node[anchor=south] at (AR.north) {\jasn};
      \tjasn{A}{B}{\B}{\1}{(0,0)}
    \end{tikzpicture},
  \end{center}
  \nrp 5
  a ``\seq'' consists of an infinite, ordered list of such \jasn{}s:
  \begin{center}
    \begin{tikzpicture}
      \node (F) at (0,0) {\tikz{
      \tjasn{A1}{B1}{\B}{\1}{(0,0)}
      \tjasn{A2}{B2}{\B}{\0}{($(A1.east) + (0.8,0)$)}
      \tjasn{A3}{B3}{\R}{\1}{($(A2.east) + (0.8,0)$)}
      \tjasn{A4}{B4}{\B}{\1}{($(A3.east) + (0.8,0)$)}
      \tjasn{A5}{B5}{\R}{\1}{($(A4.east) + (0.8,0)$)}
      \node[anchor=west] at ($(A5.south east) + (0.3,0)$) {$\cdots$};
      }};
      \node[anchor=south] at (F.north) {\seq};
    \end{tikzpicture},
  \end{center}
  and a ``\trl'' denotes a particular item (i.e. an indexed \jasn) of this \seq.

  \nrp 6
  Now, suppose something in particular about this \seq:
  if a \jasn\ occurs at any \trl, it will occur in infinitely many \trl{}s,
  and, additionally, as we the traverse \seq\ in order, the ratio of occurrences of the
  \jasn\ to \trl{}s so far becomes arbitrarily close to 
  some positive number. 
  Otherwise, if an \jasn\ never occurs, the ratio remains at zero.
  Let's call this ratio the ``\sprb'' of the \jasn.
  Let's refer to such a \seq\ as a \pseq.

  \newpage
  \nrp 7
  For example, if the \seq\ above repeated the first five \trl{}s, it would be a \pseq\ defining the \sprbs:
  \begin{center}
    \begin{tikzpicture}
      \node (F) at (0,0) {\tikz{
        \tjasn{A1}{B1}{\B}{\1}{(0,0)}
        \tjasn{A2}{B2}{\B}{\0}{($(A1.east) + (2,0)$)}
        \tjasn{A3}{B3}{\R}{\1}{($(A2.east) + (2,0)$)}
        \tjasn{A4}{B4}{\R}{\0}{($(A3.east) + (2,0)$)}

        \node[anchor=south] at (A1.north) {0.4};
        \node[anchor=south] at (A2.north) {0.2};
        \node[anchor=south] at (A3.north) {0.4};
        \node[anchor=south] at (A4.north) {0.0};
        }};
      \node[anchor=south] at (F.north) {\sprbs};
    \end{tikzpicture}
  \end{center}

  \nrp 8
  Let a ``\dist'' be a set of \pseq{}s with the same 
  \sprb\ for every \asn, which,
  in the context of a \dist, we will simply refer to as the ``\jprb'' (``\prb'' for short)
  of that \asn.
\end{part}

\begin{part}{Notation}
  \reflnenv{definition}{xyz}{Define}{ \xyzdef.}

  \reflnenv{definition}{xyzv}{Define}{ \xyzvdef.}

  \reflnenv{definition}{match}{Let the notation}{ \matchdef.}
  
  \reflnenv{definition}{spt}{Let the notation}{ \sptdef.}
  
  \reflnenv{definition}{asn}{Let}{, \asndef.}
  
  \reflnenv{definition}{seq}{Let}{, \seqdef.}
  
  \reflnenv{definition}{seqi}{Use}{, \seqidef.}
  
  \reflnenv{definition}{seqim}{Let}{, \seqimdef.}
  
  \reflnenv{definition}{sqprob}{Define}{, \prsqdef.}
  
  \reflnenv{definition}{sqmprob}{Define}{, \mprsqdef.}
  
  \reflnenv{definition}{sqcprob}{Define}{, \cprsqdef.}
  
  \reflnenv{definition}{seqp}{Let}{, \seqpdef.}
\end{part}

\begin{part}{Propositions}
  \reflnenv{proof}{sqprobdef}{It can be shown}{ that, \sqpproblimdefprop.}
  
  \reflnenv{proof}{sqmprobexpr}{It can be shown}{ that, \mprsqexprprop.}

  \reflnenv{proof}{sqmprobgexpr}{It can be shown}{ that, \sqpmprobgexprprop.}
  
  \reflnenv{proof}{sqmprobdef}{It can be shown}{ that, \sqpmproblimdefprop.}

  \reflnenv{proof}{sqmargzimpl}{It can be shown}{ that, \sqpmargzimplprop.}

  \reflnenv{proof}{sqmarggimpl}{It can be shown}{ that, \sqpmarggimplprop.}

  \reflnenv{proof}{seqimzo}{It can be shown}{ that, \seqimzoprop.}

  \reflnenv{proof}{seqimimpl}{It can be shown}{ that, \sequimimplprop.}
  
  \reflnenv{proof}{sqcprobexpr}{It can be shown}{ that, \cprsqexprprop.}

  \reflnenv{proof}{sqcprobdef}{It can be shown}{ that, \sqcpproblimdefprop.}

  \reflnenv{proof}{sqcprobdefimpl}{It can be shown}{ that, \sqpcpproblimdefimplprop.}

  \reflnenv{proof}{sqcprobtomprob}{It can be shown}{ that, \sqpcprobtomprobprop.}

  \reflnenv{proof}{sqponecind}{It can be shown}{ that, \sqpponecindprop.}
  
  \reflnenv{proof}{sqpzerocind}{It can be shown}{ that, \sqppzerocindprop.}

  \reflnenv{proof}{sqcpmprobgexpr}{It can be shown}{ that, \sqcpmprobgexprprop.}

  \reflnenv{proof}{sqcpcprobexpr}{It can be shown}{ that, \sqcpcprobexprprop.}

  \reflnenv{proof}{sqcpmargzimpl}{It can be shown}{ that, \sqcpmargzimplprop.}

  \reflnenv{proof}{sqcpmarggimpl}{It can be shown}{ that, \sqcpmarggimplprop.}

  \reflnenv{proof}{sqcpcprobtomprob}{It can be shown}{ that, \sqcpcprobtomprobprop.}
\end{part}

\begin{part} {Notation}
  \reflnenv{definition}{dist}{Define}{, \distdef.}
  
  \reflnenv{definition}{prob}{Define}{, \probdef.}
  
  \reflnenv{definition}{mprob}{Define}{, \mprobdef.}
  
  \reflnenv{definition}{cprob}{Define}{, \cprobdef.}%
\end{part}

\begin{part}{Notation}
  \reflnenv{definition}{sqev}{Define}{, \evsqdef.}%
  
  \reflnenv{definition}{sqevc}{Define}{, \evsqcdef.}%
\end{part}

\begin{part} {Propositions}
  \reflnenv{proof}{sqevexpr}{It can be shown}{ that, \evsqexprprop.}
  
  \reflnenv{proof}{sqevcexpr}{It can be shown}{ that, \evsqcexprprop.}

  \reflnenv{proof}{sqevctoev}{It can be shown}{ that, \sqpevctoevexprprop.}
\end{part}

\begin{part} {Notation}
  \reflnenv{definition}{ev}{Define}{, \evdef.}%

  \reflnenv{definition}{evc}{Define}{, \evcdef.}%
\end{part}

\begin{part} {Notation}
  \reflnenv{definition}{ind}{Define}{, \inddef.}%
\end{part}

\begin{part} {Propositions}
  \reflnenv{proof}{indprod}{It can be shown}{ that, \indproddef.}

  % TODO
  \reflnenv{proof}{indsymm}{It can be shown}{ that, \indsymmdef.}
\end{part}

\begin{part}{Topics}
  \refln{topic}{rl}{Reinforcement Learning Theory}
\end{part}
