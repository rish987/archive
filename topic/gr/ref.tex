\begin{center}
\nrp 1
{\LARGE \hlc[___gray]{Graph Theory}}\\
\end{center}
\begin{part}{Physical intuition}
  \nrp 2
  Suppose that we have a ``\gr'' consisting of a set of ``\vtx{}s''
  and a set of ``\edg{}s'' between the \vtx{}s:
  \def\A{\rm A}
  \def\B{\rm B}
  \def\C{\rm C}
  \def\D{\rm D}
  \begin{center}
    \begin{tikzpicture}
      \node[draw=black, fill=blue!70,inner sep=3mm,circle] (A) at (0,0) {\A};
      \node[draw=black, fill=blue!70,inner sep=3mm,circle] (B) at (5,0) {\B};
      \node[draw=black, fill=blue!70,inner sep=3mm,circle] (C) at (5,-5) {\C};
      \node[draw=black, fill=blue!70,inner sep=3mm,circle] (D) at (0,-5) {\D};
      \draw[-{>[length=3mm,width=3mm]}, thick] (A) -- (B);
      \draw[-{>[length=3mm,width=3mm]}, thick] (B) -- (C);
      \draw[-{>[length=3mm,width=3mm]}, thick] (C) -- (D);
      \draw[-{>[length=3mm,width=3mm]}, thick] (D) -- (A);
    \end{tikzpicture}.
  \end{center}
  \nrp 3
  Here, our \gr\ has the \vtx{}s  $\{\rm \A, \B, \C, \D\}$ and \edg{}s
  $\{ 
  \OS{\A, \B}, 
  \OS{\B, \C}, 
  \OS{\C, \D}, 
  \OS{\D, \A}
  \}$.
\end{part}
\begin{part}{Notation}
  \reflnenv{definition}{node}{Define}{ \nodedef.}
  \reflnenv{definition}{edge}{Define}{ \edgedef.}
  \reflnenv{definition}{graph}{Define}{ \graphdef.}
  \reflnenv{definition}{graphsub}{Define}{ \graphsubdef.}
  \reflnenv{definition}{allpaths}{Define}{ \allpathsdef.}
  \reflnenv{definition}{allpathseq}{Define}{ \allpathseqdef.}
  \reflnenv{definition}{paths}{Define}{ \pathsdef.}
  \reflnenv{definition}{pathssub}{Define}{ \pathssubdef.}
  \reflnenv{definition}{path}{Define}{ \pathdef.}
  \reflnenv{definition}{pathsub}{Define}{ \pathsubdef.}
  \reflnenv{definition}{spaths}{Define}{ \spathsdef.}
  \reflnenv{definition}{spathssub}{Define}{ \spathssubdef.}
  \reflnenv{definition}{spath}{Define}{ \spathdef.}
  \reflnenv{definition}{spathsub}{Define}{ \spathsubdef.}
\end{part}
\begin{part} {Propositions}
  \reflnenv{proof}{subpath}{It can be shown}{ that, \subpathprop.}
  \reflnenv{proof}{subpathcomb}{It can be shown}{ that, \subpathcombprop.}
  \reflnenv{proof}{pathuniq}{It can be shown}{ that, \pathuniqprop.}
  \reflnenv{proof}{pathswap}{It can be shown}{ that, \pathswapprop.}
\end{part}
\begin{part}{Topics}
  \refln{topic}{pr}{Probability Theory}
\end{part}
