\begin{center}
  \nrp
  {\LARGE \hlc[___gray]{The Archive}}\\
  \vspace{10pt}
  \large 
  \href{https://rish987.github.io/}{\tt {rish987.github.io}}\\
  \vspace{7pt}
  \today
\end{center}
\begin{part}{Introduction}
  \nrp
  This is my personal archive,
  a place where I plan to record
  what I've learned in the fields of 
  mathematics, computer science, and more.

  A continual work in progress, the intent is 
  for it to be a completely self-contained repository,
  with descriptions that are as concise as physically possible.

  The archive is free and open-source, 
  together with the archiver, the tool I use to build archives.
  You can find them below:

  \href{https://github.com/rish987/archive}{\tt github.com/rish987/archive}

  \href{https://github.com/rish987/archiver}{\tt github.com/rish987/archiver}.

\end{part}

\begin{part}{Navigation}
  \nrp
  It's organized hierarchically into units called ``\refr{}s,''
  which can be one of the following types:
  \begin{itemize}
    \item \nrp ``\refroot.'' The \refr\ you're currently on!
      There's only one \refr\ of this type. 
      It's reserved for descriptions that apply to the entire archive.
    \item \nrp ``\tpc{}'' (denoted by \topicd). These repeat a general format:
    \begin{itemize}
      \item Describe a simple concept that relies on a simple physical intuition.
      \item Correspond this intuition with some initial notation, and 
        possibly define some more notation using this initial notation.
      \item Use this notation to describe some facts that follow from it.
        These facts are merely stated, not proven.
    \end{itemize}
    \item \nrp ``\defn'' (denoted by \definitiond). 
      These introduce notation that depend on a previously described
      physical intuition, previously defined notation, or a combination of both.
    \item \nrp ``\proofl{}'' (denoted by \proofd). 
      These state a fact and show why it is true, using previously defined notation.
    \item \nrp ``\notel{}'' (denoted by \noted). 
      These may be one of the following:
      \begin{itemize}
        \item A visual/verbal interpretation of a concept that isn't strictly 
          necessary for understanding, but could be helpful.
        \item A temporary stand-in for a \refr\ that will be properly integrated 
          into its own topic at a later time.
      \end{itemize}
  \end{itemize}

  \nrp
  Each \refr\ is delimited by a pair of horizontal rules.

  At the upper-left corner of each \refr\ is a ``\refrnum'' (``RN'' for short),
  a unique ID that will never, ever change. 
  When verbally referring to a particular \refr, we say ``RN'' followed by 
  the \refrnum, e.g. ``RN53.'' 
  The numbers themselves don't actually mean anything,
  other than roughly tracking the order in which I created the \refr{}s.
  In fact, some \refrnum{}s may not actually exist, if I decided to delete its former \refr!

  At the upper-right corner of each \refr\ is a ``\refrpat,'' 
  a unique path that identifies this \refr's place in the hierarchy of \refr{}s.
  It's written in UNIX-SHELL path style.

  \nrp
  A \refr\ is linked to either explicitly or implicitly.
  These links may either be ``local'' or ``global.'' 
  The difference is contextual -- the 
  first link to a \refr\ in the hierarchy
  is a ``local'' link, in that following it will append to the current \refrpat.
  All other links are ``global,'' in that the current \refrpat\ has
  no relation to that of the linked \refr.
  \begin{itemize}
    \item \nrp An explicit link has the form \exampleln.
    \item \nrp An implicit link is present in the notation, 
      and links to the \refr\ where the notation was originally defined.
      Just as notation is nested, so too are the corresponding implicit links.

      We follow the convention that implicit links are always global;
      it follows that the first link to a \refr\ is always explicit.
  \end{itemize}
\end{part}

\begin{part}{Annotations}
  \nrp
  Equations may be annotated as, for example:
  $$a \rellr = x i b \rell = y c \rellr {\ge} z j d,$$
  followed by
  $$g \relr = {(x-z)->\exampleln} h,$$
  where blue underset boxes define ``\labell{}s,''
  and grey overset boxes provide ``\explnl{}s'' that explain why an (in)equality follows, 
  using previously defined \labell{}s and other \refr{}s.
  An \explnl\ is formatted as follows:
  \begin{itemize}
    \item \nrp {\tt x} indicates to consider the single (in)equality labeled by {\tt x}.

      In the example above, {\tt x} refers to $a = b$.
    \item \nrp {\tt (x-y)} indicates that, when {\tt x} and {\tt y} appear in a chain (in)equality,
      to consider the equation that results from cropping out what appears 
      in between the (in)equalities labeled by {\tt x} and {\tt y}.

      In the example above, {\tt(x-z)} refers to $a \ge d$.
    \item \nrp {\tt x->\exampleln} indicates to use (in)equality {\tt x}
      as the antecendent of the \refr\ linked to by \exampleln.

      In the example above, {\tt(x-z)->\exampleln} refers to use $a \ge d$
      as an antecedent of the \refr\ with RN {\tt[RN]}.
  \end{itemize}
\end{part}

\begin{part}{Formats}
  \nrp
  The archive is available in the following formats:
  \begin{itemize}
    \item \nrp \href{https://rish987.github.io/files/archives/tree_online/archive/tree_online.pdf}{\tt {tree\_online}}:
      Each \refr\ has its own online link in a ``\refr\ tree'' and corresponding PDF. 
      
      Links are online hyperlinks.
    \item \nrp \href{https://rish987.github.io/files/archives/tree/archive.tar.gz}{\tt {tree}}:
      Local download of the \refr\ tree.

      Links are remote PDF links.
    \item \nrp \href{https://rish987.github.io/files/archives/full/archive/full.pdf}{\tt {full}}:
      All \refr{}s are combined into a single PDF, ordered by depth (i.e., a linearized DAG),
      where each \refr\ starts on a new page.

      Links are cross-referencing links within the same PDF.
    \item \nrp \href{https://rish987.github.io/files/archives/full_compact/archive/full_compact.pdf}{\tt {full\_compact}}:
      Identical to {\tt full}, except \refr{}s may start on the same page.

      Links are cross-referencing links within the same PDF.
  \end{itemize}

  \nrp
  For the best experience, I highly recommend using \href{https://mozilla.github.io/pdf.js/}{\tt pdf.js} 
  (native to the \href{https://www.mozilla.org/en-US/firefox/new/}{Firefox web browser},
  but also \href{https://chrome.google.com/webstore/detail/pdf-viewer/oemmndcbldboiebfnladdacbdfmadadm}{available on Chrome})
  as your PDF viewer, in ``presentation mode'' ({\tt ctrl+alt+p}) on a vertical monitor.
  Note that with {\tt pdf.js}, you can use your browser's history navigation functionality
  to traverse your link history ({\tt alt+rightarrow} and {\tt alt+leftarrow}).

  \nrp
  For offline viewing, I highly recommend 
  \href{http://manpages.ubuntu.com/manpages/xenial/man1/atril.1.html}{Atril Document Viewer}.
\end{part}

\begin{part}{Topics}
\refln{topic}{prob}{Probability Theory}
\end{part}
