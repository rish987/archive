\begin{proposition}
  We have that \pexprtprop{.}%
\end{proposition}

\begin{proof}
  \hrule
  {\it Base case.}
  When $\TT = 1$,
  we have \reflng{definition}{archives/topic/rl_theory/definition/ind}{by definition} that
  $\bca = \bcind$.

  \serule
  {\it Inductive hypothesis.}
  Assume that for $\TT - 1$ such that $1 \le \TT - 1 \le \T - 1$,
  the proposition holds. That is,
  $$ \pexprtihp \rell = ih \pexprtih.$$%
  \srule
  {\it Inductive step.}
  We want to show that the proposition holds for $\TT.$

  There are two cases to consider:
  \begin{itemize}
    \item Case 1: $\pexprtihp \rell = co 0$. 
      Then,
      $$\pexprtihp \relr = co \ 0\  \rellrf{=}{coz}{\mrefmargzimpl} \pexprtp,$$
      and
      \begin{align*}
        &\pexprt \\
        =\ &\pexprtih \times \\
        &\isact \times \istrd\\
        \relr = ih \ & \pexprtihp \times \isact \times \istrd \\
        \relr = co \ & 0 \relr = coz \pexprtp.
      \end{align*}
    \item Case 2: $\pexprtihp \rell > ct 0$. 
      Then, 
      \ctext{$\isbf \rellrf{\relis}{daf}{ct,\mrefcondprob} \defined$, $\isb \rellrf{\relis}{da}{ct,\reflnm{proof}{marggimpl},\mrefcondprob} \defined$,}
      and we have
      $$\isbf \rellr = cia daf,\mreflng{proof}{archives/topic/rl_theory/proof/actcind} \isb 
      \rellr = ad da,\mreflng{definition}{archives/topic/rl_theory/definition/pact} \isact.$$

      \reflng{definition}{archives/topic/rl_theory/definition/pactdef}{There are two cases} to consider:
      \begin{itemize}
        \item Case 2.1: $\isact \rell = cto 1$.
          Then,
          \begin{align*}
            \ismid &\rellr = md daf \isbf \times \pexprtihp\\
                   &\relr = cia-ad \isact \times \pexprtihp\\
                   &\relr = cto \pexprtihp\\
                   &\rellr > mg ct 0.
          \end{align*}
          Therefore, 
          $$\iscf \rellrf{\relis}{dsf}{md-mg,\mrefcondprob} \defined,\isc \rellrf{\relis}{ds}{md-mg,\mrefmarggimpl,\mrefcondprob} \defined$$
          and we have
          $$\iscf 
          \rellr = cis dsf,\mreflng{definition}{archives/topic/rl_theory/definition/trdcind} \isc 
          \rellr = sd ds,\mreflng{definition}{archives/topic/rl_theory/definition/trd} \istrd.$$
          Therefore,
          \begin{align*}
            \pexprtp 
            \relr = daf,dsf \ & \pexprtihp \times \isbf \times\\& \iscf\\
            \relr = cia,cis \ & \pexprtihp \times \isb \times \isc\\
            \relr = ih,ad,sd \ & \pexprg{\TT}{\TT'}.
          \end{align*}
        \item Case 2.2: $\isact \rell = ctt 0$.
          Then,
          \begin{align*}
            \ismid &\relr = daf \isbf \times \pexprtihp\\
                   &\relr = cia-ad \isact \times \pexprtihp\\
                   &\relr = ctt 0.
          \end{align*}
          Therefore,
          $$\pexprtp \rell = cttz 0,$$
          so
          $$\pexprtp
          \relr = cttz \ 0 \ 
          \relr = ctt \pexprt.$$
      \end{itemize}
  \end{itemize}
  \hrule
\end{proof}
