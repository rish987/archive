\begin{proposition}
  We have that \pexprtprop{.}%
\end{proposition}

\begin{proof}
  \hrule
  {\it Base case.}
  When $\TT = 1$,
  we have \reflng{definition}{archives/topic/rl_theory/definition/ind}{by definition} that
  $\bca = \bcind$.

  \srule
  {\it Inductive hypothesis.}
  Assume that for $\TT - 1$ such that $1 \le \TT - 1 \le \T - 1$,
  the proposition holds. That is,
  $$ \pexprjointdistg{\TT - 1} = \pexprgg{\TT - 1}{\TT - 2}{\TT'}.$$%
  \srule
  {\it Inductive step.}
  We want to show that the proposition holds for $\TT.$

  There are two cases to consider:
  \begin{itemize}
    \item Case 1: $\isa = 0$. 
      Then,
      $$\isa = 0 = \isjoint,$$
      and
      \begin{align*}
      &\pexprg{\TT}{\TT'} \\
      =\ &\pexprgg{\TT - 1}{\TT - 2}{\TT'} \times \\
      &\isact \times \istrd\\
      \noteop i = \ & \isa \times \isact \times \istrd \\
      \noteop ii = \ & 0 = \isjoint,
      \end{align*}
      where (i) follows from the inductive hypothesis,
      and (ii) follows from the fact that $\isa = 0$.
    \item Case 2: $\isa > 0$. 
      Then, 
      \ctext{$\isbf$ and $\isb$ are defined,}
      and we have
      $$\isbf \noteop i = \isb \noteop ii = \isact,$$
      where (i) is \reflng{proof}{archives/topic/rl_theory/proof/actcind}{implied by definition}, and (ii) \reflng{definition}{archives/topic/rl_theory/definition/pact}{follows by definition}.

      \reflng{definition}{archives/topic/rl_theory/definition/pactdef}{There are two cases} to consider:
      \begin{itemize}
        \item Case 2.1: $\isact = 1$.
          Then,
          \begin{align*}
            \ismid &= \isbf \times \isa\\
                   &= \isact \times \isa\\
                   &= \isa\\
                   &> 0.
          \end{align*}
          Therefore, 
          \ctext{$\iscf$ is defined,}
          and we have
          $$\iscf \noteop i = \isc \noteop ii = \istrd.$$
          where \reflng{definition}{archives/topic/rl_theory/definition/trdcind}{(i)} and 
          \reflng{definition}{archives/topic/rl_theory/definition/trd}{(ii)} follow by definition.
          Therefore,
          \begin{align*}
            \isjoint =\ & \isa \times \isbf \times\\& \iscf\\
            =\ & \isa \times \isb \times \isc\\
            =\ & \pexprg{\TT}{\TT'}.
          \end{align*}

        \item Case 2.2: $\isact = 0$.
          Then,
          \begin{align*}
            \ismid &= \isbf \times \isa\\
                   &= \isact \times \isa\\
                   &= 0.
          \end{align*}
          Therefore,
          $$\isjoint = 0,$$
          so
          $$\isjoint = \pexprg{\TT}{\TT'} = 0.$$
      \end{itemize}
  \end{itemize}
\end{proof}
