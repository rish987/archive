\begin{proposition}
  For any $1 \le \TT \le \T$, for
  \ctext{\pexprdomg\TT} and for \ctext{\pexprdompol,} 
  we can write
  $$\pexprjointdistasng{\TT}$$ as
  $$ \pexprjointdistg{\TT} = \pexprg{\TT}{\TT'}.$$%
\end{proposition}

\begin{proof}
  \def\last{\TT - 1}
  \def\sslast{\SS_{\last}}
  \def\aalast{\AA_{\last}}
  \def\prelist{\SS_1, \AA_1, \SS_2, \ldots, \sslast}
  \def\postlist{\SS_1, \AA_1, \SS_2, \ldots, \SS_\TT}
  \def\pprelist{\PSTT(\prelist)}
  \def\isact{\PACT{\STT}{\last}{\aalast}{\sslast}}
  \def\istrd{\PTRD{\SS_\TT}{\aalast}{\sslast}}
  \def\isjoint{\PSTT(\postlist)}
  \def\isa{\pprelist}
  \def\isbf{\PSTT(\aalast \mid \prelist)}
  \def\isb{\PSTT(\aalast \mid \sslast)}
  \def\ismid{\PSTT(\prelist, \aalast)}
  \def\iscf{\PSTT(\SS_\TT \mid \aalast, \prelist)}
  \def\isc{\PSTT(\SS_\TT \mid \AA_{\TT - 1}, \SS_{\TT - 1})}
  \def\bcind{\PIND(\SS_1)}
  \def\bca{\PSTT(\SS_1)}
  \hrule
  {\it Base case.}
  When $\TT = 1$,
  we have by definition that
  $\bca = \bcind$.

  \srule
  {\it Inductive hypothesis.}
  Assume that for $\TT - 1$ such that $1 \le \TT - 1 \le \T - 1$,
  the proposition holds. That is,
  $$ \pexprjointdistg{\TT - 1} = \pexprg{\TT - 1}{\TT'}.$$%

  \srule
  {\it Inductive step.}
  We want to show that the proposition holds for $\TT.$

  We can consider two cases:
  \begin{itemize}
    \item Case 1: $\isa = 0$. 
      Then,
      $$\isa = 0 = \isjoint,$$
      and by the induction hypothesis,
      \begin{align*}
      &\pexprg{\TT}{\TT'} \\
      &= \isa \times \isact \times \istrd \\
      &= 0 = \isjoint.
      \end{align*}
    \item Case 2: $\isa > 0$. 
      Then, 
      \ctext{$\isbf$ and $\isb$ are defined,}
      and we have
      $$\isbf = \isb = \isact,$$
      with $$\isa \times \isbf = \ismid.$$

      There are two cases to consider:
      \begin{itemize}
        \item Case 2.1: $\ismid = \isa > 0$.
          Then, 
          \ctext{$\iscf$ is defined,}
          and we have
          $$\iscf = \isc = \istrd.$$
          Therefore,
          \begin{align*}
            \isjoint =\ & \ismid \times \iscf\\
                     =\ & \isa \times \isbf \times\\& \iscf\\
            =\ & \isa \times \isb \times \isc\\
            =\ & \pexprg{\TT}{\TT'}.
          \end{align*}
        \item Case 2.2: $\ismid = 0$.
          Then,
          $$\isjoint = 0$$
          and
          $$\isact = 0$$,
          so
          $$\isjoint = \pexprg{\TT}{\TT'} = 0.$$
      \end{itemize}
  \end{itemize}
\end{proof}
