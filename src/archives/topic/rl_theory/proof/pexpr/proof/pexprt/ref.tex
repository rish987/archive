\begin{proposition}
  For any $1 \le \TT \le \T$, for
  \ctext{\pexprdomg\TT} and for \ctext{\pexprdompol,} 
  we can write
  $$\pexprjointdistasng{\TT}$$ as
  $$ \pexprjointdistg{\TT} = \pexprg{\TT}{\TT'}.$$%
\end{proposition}

\begin{proof}
  \hrule
  {\it Base case.}

  When $\TT = 2$, we can consider two cases:
  \begin{itemize}
    \def\bcact{\PACT{\STT}{1}{\AA_1}{\SS_1}}
    \def\bctrd{\PTRD{\SS_2}{\AA_1}{\SS_1}}
    \def\bcind{\PIND(\SS_1)}
    \def\bcjoint{\P(\SS_1, \AA_1, \SS_2)}
    \def\bca{\PSTT(\SS_1)}
    \def\bcb{\PSTT(\AA_1 \mid \SS_1)}
    \def\bcc{\PSTT(\SS_2 \mid \AA_1, \SS_1)}
    \item Case 1: $\bcind = 0$. 
      By definition, we have
      $$\bca = \bcind = 0,$$
      and so
      $$\bca = 0 = \bcjoint,$$
      and
      $$\bcjoint = \bcind\bcact\bctrd = 0.$$
    \item Case 2: $\bcind > 0$. 
      By definition, we have
      $$\bca = \bcind > 0$$
      and so 
      \lnproof{whendefpa}{$\bcb$ is defined}
      and 
      \lnproof{whendefps}{$\bcc$ is defined}.
      By definition,
      $$\bcb = \bcact$$
      and $$\bcc = \bctrd.$$
      Therefore,
      $$\bcjoint = \bca\bcb\bcc = \bcind\bcact\bctrd.$$
  \end{itemize}
\end{proof}
