\begin{proposition}
  We have that \pexprtprop{.}%
\end{proposition}

\begin{proof}
  \hrule
  {\it Base case.}
  When $\TT = 1$,
  we have \reflng{definition}{archives/topic/rl_theory/definition/ind}{by definition} that
  $\bca = \bcind$.

  \srule
  {\it Inductive hypothesis.}
  Assume that for $\TT - 1$ such that $1 \le \TT - 1 \le \T - 1$,
  the proposition holds. That is,
  $$ \pexprtihp = \pexprtih.$$%
  \srule
  {\it Inductive step.}
  We want to show that the proposition holds for $\TT.$

  There are two cases to consider:
  \begin{itemize}
    \item Case 1: $\pexprtihp = 0$. 
      Then,
      $$\pexprtihp = 0 = \pexprtp,$$
      and
      \begin{align*}
        &\pexprt \\
        =\ &\pexprtih \times \\
        &\isact \times \istrd\\
        \noteop i = \ & \pexprtihp \times \isact \times \istrd \\
        \noteop ii = \ & 0 = \pexprtp,
      \end{align*}
      where (i) follows from the inductive hypothesis,
      and (ii) follows from the fact that $\pexprtihp = 0$.
    \item Case 2: $\pexprtihp > 0$. 
      Then, 
      \ctext{$\isbf$ and $\isb$ are defined,}
      and we have
      $$\isbf \noteop i = \isb \noteop ii = \isact,$$
      where (i) is \reflng{proof}{archives/topic/rl_theory/proof/actcind}{implied by definition}, and (ii) \reflng{definition}{archives/topic/rl_theory/definition/pact}{follows by definition}.

      \reflng{definition}{archives/topic/rl_theory/definition/pactdef}{There are two cases} to consider:
      \begin{itemize}
        \item Case 2.1: $\isact = 1$.
          Then,
          \begin{align*}
            \ismid &\noteop i = \isbf \times \pexprtihp\\
                   &\noteop ii = \isact \times \pexprtihp\\
                   &\noteop iii = \pexprtihp\\
                   &> 0,
          \end{align*}
          where 
          (i) follows from \todo,
          (ii) follows from \todo, 
          and
          (iii) follows from \todo.
          Therefore, 
          \ctext{$\iscf$ is defined,}
          and we have
          $$\iscf \noteop i = \isc \noteop ii = \istrd.$$
          where \reflng{definition}{archives/topic/rl_theory/definition/trdcind}{(i)} and 
          \reflng{definition}{archives/topic/rl_theory/definition/trd}{(ii)} follow by definition.
          Therefore,
          \begin{align*}
            \pexprtp \noteop i =\ & \pexprtihp \times \isbf \times\\& \iscf\\
            \noteop ii =\ & \pexprtihp \times \isb \times \isc\\
            \noteop iii =\ & \pexprg{\TT}{\TT'},
          \end{align*}
          where 
          (i) follows from \todo,
          (ii) follows from \todo,
          and
          (iii) follows from \todo.

        \item Case 2.2: $\isact = 0$.
          Then,
          \begin{align*}
            \ismid &\noteop i = \isbf \times \pexprtihp\\
                   &\noteop ii = \isact \times \pexprtihp\\
                   &\noteop iii = 0,
          \end{align*}
          where 
          (i) follows from \todo,
          (ii) follows from \todo,
          and
          (iii) follows from \todo.
          Therefore,
          $$\pexprtp = 0,$$
          so
          $$\pexprtp \noteop i = \pexprt \noteop ii = 0,$$
          where
          (i) follows from \todo\ 
          and
          (ii) follows from \todo.
      \end{itemize}
  \end{itemize}
\end{proof}
