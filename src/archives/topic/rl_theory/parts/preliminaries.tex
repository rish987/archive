\begin{part}{Preliminaries}
  Let's now describe some preliminary results using this notation.

  \begin{refenv}{proof}{pexpr}
    One of the first questions we may ask is, does our notation completely describe
    the distribution of \run{}s for a particular \str?
    That is, 
    \ctext{for \pexprdom,}
    and
    \ctext{for \pexprdompol,}
    can we determine the value of
    $$\pexprjointdistasn?$$
    The answer to this question is: yes! 
    We have all the information we need 
    to completely describe 
    this joint distribution. 
    \refln{proof}{pexpr}{It can be written} as
    \begin{align*}
      \text{\refln{note}{implasn}{$\pexprjointdist$}}
      =\ &\PIND{\SS_1}\times\\
         &\PACT{\STT}{1}{\AA_1}{\SS_1}
          \PTRD{\SS_2}{\AA_1}{\SS_1}\times\\
         &\PACT{\STT}{2}{\AA_2}{\SS_2}
          \PTRD{\SS_3}{\AA_2}{\SS_2}\times\\
         &\ldots\times\\
         &\PACT{\STT}{{\T - 1}}{\AA_{\T - 1}}{\SS_{\T - 1}}
          \PTRD{\SS_\T}{\AA_{\T - 1}}{\SS_{\T - 1}},
    \end{align*}
    or, in more compact \refln{note}{prodnot}{product notation},
    $$\pexprjointdist = \pexpr.$$
  \end{refenv}
\end{part}
