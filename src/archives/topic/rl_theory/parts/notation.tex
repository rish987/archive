\begin{part}{Notation}
  To answer this question, let's establish some notation.

  Let's \lndefinition{states}{define}
  \statesdef{,}
  and \lndefinition{actions}{define}
  \actionsdef{,}
  where $\S$ and $\A$ are finite.
  \lndefinition{rewards}{Define}
  \rewardsdefg{\lnnote{fnnot}{$\rewardsdefnot$} $\defeq$}{.}

  \lndefinition{steplim}{Define}
  \steplimdef{,}
  \lndefinition{str}{define}
  \strdef{,}
  and \lndefinition{strs}{define}
  \strsdef{.}
  Note that \lnproof{npol}{$|\ST| = \T \times |\S| \times |\A|$}.

  \lndefinition{inddef}{Define} \inddefdefg{\lnnote{paramprob}{$\inddefp$} $\defeq$}{.}
  \lndefinition{pactdef}{Define} \pactdefdef{.}
  \lndefinition{trddef}{Define} \trddefdef{.}

  Now, suppose the \agt\ is following some \str\ $\STT$.
  \lndefinition{rvst}{Define} \rvstdef{,}
  and \lndefinition{rvat}{define} 
  \rvatdef{,}
  whose distributions are dependent on $\STT$.

  Let $\PSTT(\cdot)$ and $\ESTT[\cdot]$ denote the probability of an event
  and the expectation of a quantity
  when the \agt\ follows $\STT$.
  Because we follow a fixed \str, \lndefinition{pact}{we know that}, \pactdeffull{,}
  \lnproof{actcind}{from which it follows} that
  $\AV_\TT$ is independent of all previous \til{}s and \act{}s given $\SV_\TT$.
  Notationally, \actcindpropfullg{\ctext{\lnnote{implgen}{$\actcindprop$}.}}
  \lndefinition{ind}{We know that} \inddef{,}
  and
  \lndefinition{trd}{we know that} \trddef{,}
  \lndefinition{trdcind}{where} \trdcinddef{.}

  We can now write the \trwd\ as
  $$\sum_{\TT = 1}^\T \RW(\SV_\TT)$$
  and the \atrwd\ as
  $$\ESTT\left[\sum_{\TT = 1}^\T \RW(\SV_\TT)\right].$$
\end{part}
