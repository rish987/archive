\documentclass{rl_theory/note}

\begin{document}
\begin{note}
  \def\quant{\PSTT(\S_\TT = \SS \mid \A_\TTM = \STT_\TTM(\SS'), \S_\TTM = \SS')}
  This ``reachability requirement'' is necessary
  because of the fact that
  $$\text{ $\quant$ is defined } 
  \iff \PSTT(\S_{\TT - 1} = \SS') > 0.$$%

  From the definition of conditional probability, we have
  $$ \quant 
  = \frac{\PSTT(\S_\TT = \SS, \A_\TTM = \STT_\TTM(\SS'), \S_\TTM = \SS')}
    {\PSTT(\A_\TTM = \STT_\TTM(\SS'), \S_\TTM = \SS')}.$$

  In one direction, we have that if $\PSTT(\S_\TTM = \SS') > 0$, then 
  $$\PSTT(\A_\TTM = \STT_\TTM(\SS')\mid \S_\TTM = \SS') = 1$$ 
  is defined, so
  we can write the denominator as
  \begin{align*}
    \PSTT(\A_\TTM = \STT_\TTM(\SS'), \S_\TTM = \SS') &= \PSTT(\A_\TTM = \STT_\TTM(\SS')\mid \S_\TTM = \SS')
    \PSTT(\S_\TTM = \SS')\\
                                                     &= \PSTT(\S_\TTM = \SS') > 0,
  \end{align*}
  so $\quant$ is defined.

  In the other direction, we have that if $\PSTT(\S_\TTM = \SS') = 0$, then 
  $$ \PSTT(\A_\TTM = \STT_\TTM(\SS'), \S_\TTM = \SS') = 0,$$ 
  so $\quant$
  is undefined.
\end{note}
\end{document}
