\documentclass{rl_theory/rl_theory}
\usepackage{calc}
\usepackage{xintexpr}
\usetikzlibrary{calc}
\def\sc{\setcounter}
\def\nc{\newcounter}
\def\hlg#1{\hlc[lightlightgray]{#1}}

\begin{document}

\title{
\hlc[lightlightgray]{Reinforcement Learning Theory}\\
\vspace{10pt}
\large 
Rishikesh Vaishnav\\ 
\vspace{10pt}
\normalsize 
Adapted from
{\it Neuro-Dynamic Programming}
by \\
Dmitri P. Bertsekas
and
John N. Tsitsiklis.
}
\maketitle

\pgfkeys{/tikz/board/.style={y=-1cm, scale=1, every node/.style={transform shape}}}
\begin{part} {Motivating example}
  \def\agent#1#2{\draw ($(#1,#2)$) node [rectangle, draw=black, fill=blue, inner sep=0.14cm] {};}
  \def\tile#1#2{\draw ($(#1,#2)$) node [rectangle, draw=black, inner sep=1cm] {};}
  \def\tileidx#1#2#3{\draw ($(#1,#2) - (0.5, 0.5)$) node [rectangle, fill=lightlightgray, minimum size=0.4cm] {{#3}};}

  \def\actions#1#2#3#4{
    \draw (#1, #2) node (center) {};

    \xintifboolexpr {#3 > 0} { \draw [->] (center) -- +(0, -0.5); }{}
    \xintifboolexpr {#3 < 2} { \draw [->] (center) -- +(0, 0.5); }{}
    \xintifboolexpr {#4 < 2} { \draw [->] (center) -- +(0.5, 0); }{}
    \xintifboolexpr {#4 > 0} { \draw [->] (center) -- +(-0.5, 0); }{}
  }

  \def\brd{\hlg{board}}
  \def\agt{\hlg{agent}}
  \def\til{\hlg{tile}}
  \def\act{\hlg{action}}
  \def\stp{\hlg{step}}
  Consider the following ``\brd'':\\
  \def\boardopts{}
  \begin{center}
    \begin{tikzpicture}[board]
      \nc{row}
      \nc{col}
      \nc{x}
      \nc{y}
      \nc{idx}
      \foreach \row in {0,1,...,2}
      {\sc{row}{\row}

        \foreach \col in {0,1,...,2} 
        {\sc{col}{\col}
          \sc{x}{\value{col} * 2}
          \sc{y}{\value{row} * 2}
          \sc{idx}{(\value{col} + \value{row} * 3) + 1}

          \actions{\thex}{\they}{\row}{\col}

          \tileidx{\thex}{\they}{\theidx}{}
          \tile{\thex}{\they}
        }
      }
      \agent{0}{0}
    \end{tikzpicture},
  \end{center}
  where there are a total of 9 ``\til{}s'' (\tikz[board, scale=0.1]{\tileidx{0}{0}{$\cdot$}{}\tile{0}{0}}), 
  an ``\agt'' (\tikz[board, scale=0.7]{\agent{0}{0}}), and
  a set of ``\act{}s'' ($\rightarrow$) that the \agt\ can perform at every \til.

  At every ``\stp'' in time, the agent is in one of the \til{}s,
  and must take one of the \act{}s.
\end{part}
\end{document}
