\documentclass{rl_theory/rl_theory}
\usepackage{calc}
\usepackage{xintexpr}
\usetikzlibrary{calc}
\def\sc{\setcounter}
\def\nc{\newcounter}
\def\hlg#1{\hlc[lightlightlightgray]{#1}}

\begin{document}

\title{
\hlc[lightlightlightgray]{Reinforcement Learning Theory}\\
\vspace{10pt}
\large 
Rishikesh Vaishnav\\ 
\vspace{10pt}
\normalsize 
Adapted from
{\it Neuro-Dynamic Programming}
by \\
Dmitri P. Bertsekas
and
John N. Tsitsiklis.
}
\maketitle

\pgfkeys{/tikz/board/.style={y=-1cm, scale=1, every node/.style={transform shape}}}
\begin{part} {Motivating example}
  \def\agent{\draw (center) node [rectangle, draw=black, fill=blue, inner sep=0.14cm] {};}

  \def\ifbool{\xintifboolexpr}

  \def\tilecoords#1#2{\draw ($(#1 * \tilesep,#2 * \tilesep)$) node (center) {};}
  \def\tile#1{\draw (center) node (#1) [rectangle, draw=black, inner sep=1cm] {};}
  \def\tileidx#1{\draw ($(center) - (0.5, 0.5)$) node [rectangle, draw=black, minimum size=0.4cm] {{#1}};}

  \def\tileandidx#1{\tile{#1}\tileidx{#1}}

  \def\tilecenter#1#2{\draw (#1, #2) node (center) {};}
  \def\tiledir#1#2{\draw [->, #1] (center) -- +#2;}
  \def\tileup#1{\tiledir{#1}{(0, -0.5)}}
  \def\tiledown#1{\tiledir{#1}{(0, 0.5)}}
  \def\tileright#1{\tiledir{#1}{(0.5, 0)}}
  \def\tileleft#1{\tiledir{#1}{(-0.5, 0)}}

  \def\tiletransition#1#2#3#4{\draw [->, opacity=#4] (#1) -- node [#3] {#4} (#2);}

  \def\tilesep{2}

  \def\actions#1#2{
    \ifbool {#1 > 0} { \tileup{} }{}
    \ifbool {#1 < 2} { \tiledown{} }{}
    \ifbool {#2 < 2} { \tileright{} }{}
    \ifbool {#2 > 0} { \tileleft{} }{}
  }

  \def\brd{\hlg{environment}}
  \def\agt{\hlg{agent}}
  \def\til{\hlg{state}}
  \def\act{\hlg{action}}
  \def\stp{\hlg{step}}
  \def\trd{\hlg{transition distribution}}
  \def\ind{\hlg{initial distribution}}
  \def\rwd{\hlg{return}}
  \def\tileloop#1{
    \foreach \row in {0,1,...,2}
    {\sc{row}{\row}
      \foreach \col in {0,1,...,2}
      {\sc{col}{\col}
        \setidx
        #1
      }
    }
  }
  Consider the following ``\brd'':\\
  \def\boardopts{}
  \begin{center}
    \begin{tikzpicture}[board]
      \nc{row}
      \nc{col}
      \nc{idx}
      \def\setidx{\sc{idx}{(\value{col} + \value{row} * 3) + 1}}
      \tileloop 
      {
        \tilecoords{\col}{\row}
        \ifbool{\row == 0 && \col == 0}{\agent}{}
        \tileandidx{\theidx}
        \actions{\row}{\col}
      }

      \tileloop 
      {
        \expandafter\gdef\csname reward\theidx\endcsname{5}
      }

      \tileloop 
      {
        \tilecoords{\col}{\row}
        \draw ($(center) - (-0.5, 0.5)$) node [rectangle, fill=lightlightgray, minimum size=0.4cm] {{\csname reward\theidx\endcsname}};
      }
    \end{tikzpicture},
  \end{center}
  where there are a total of 9 ``\til{}s'' (\tikz[board, scale=0.1]{\tilecoords{0}{0}\tile{0}\tileidx{$\cdot$}}), 
  an ``\agt'' (\tikz[board, scale=0.7]{\agent{0}{0}}), and
  a set of ``\act{}s'' ($\rightarrow$) that the \agt\ can take at every \til.

  At every ``\stp'' in time, the agent is in one of the \til{}s,
  and must take one of the \act{}s. 

  At any \stp, if the \agt\ takes an \act\ in a \til, 
  it will be in a \til\ at the next \stp\ according to a ``\trd''
  (that only depends on the current \til\ and \act, not the current \stp).

  For example, the \agt\ could
  \begin{itemize}
    \def\tilesep{3.2}
    \def\tscale{0.6}
    \item take \act\ $\uparrow$ at \til\ $7$:
      \begin{center}
        \begin{tikzpicture}[board, scale=\tscale]
          \tilecoords{0}{1}
          \tileandidx{7}
          \tileup{draw=blue}
          \agent

          \tilecoords{0}{0}
          \tileandidx{4}

          \tilecoords{1}{1}
          \tileandidx{8}

          \tiletransition{7}{8}{below}{0.25}
          \tiletransition{7}{4}{left}{0.75}
        \end{tikzpicture},
      \end{center}
      and
    \item take \act\ $\rightarrow$ at \til\ $5$:
      \begin{center}
        \begin{tikzpicture}[board, scale=\tscale]
          \tilecoords{1}{1}
          \tileandidx{5}
          \tileright{draw=blue}
          \agent

          \tilecoords{0}{1}
          \tileandidx{4}

          \tilecoords{2}{1}
          \tileandidx{6}

          \tilecoords{1}{0}
          \tileandidx{2}

          \tilecoords{1}{2}
          \tileandidx{8}

          \tiletransition{5}{6}{below}{0.75}
          \tiletransition{5}{4}{above}{0.05}
          \tiletransition{5}{2}{left}{0.10}
          \tiletransition{5}{8}{right}{0.10}
        \end{tikzpicture},
      \end{center}
  \end{itemize}
  showing the probability of being in another \til\ at the next \stp\ according to the \trd.

  However, this begs the question, what \til\ is the \agt\ at initially (\stp\ 1)? 
  To decide this, there is also an \ind\ over the \til{}s.

  \newpage

  Now, let's make things a bit more interesting. 

  Suppose every \til\ also has a \rwd\ that the \agt\ recieves from it, leaving our \brd\ looking something like:
\end{part}
\end{document}
