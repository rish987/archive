\documentclass{rl_theory}

\def\brd{\hlg{environment}}
\def\agt{\hlg{agent}}
\def\til{\hlg{state}}
\def\act{\hlg{action}}
\def\stp{\hlg{step}}
\def\lmt{\hlg{step limit}}
\def\trd{\hlg{transition distribution}}
\def\ind{\hlg{initial distribution}}
\def\rwd{\hlg{return}}
\def\str{\hlg{strategy}}
\def\run{\hlg{run}}
\def\trwd{\hlg{total return}}
\def\atrwd{\hlg{average total return}}

\begin{part}{Notation}
  To answer this question, let's establish some notation.

  Let's \lndefinition{states}{define}
  \statesdef{,}
  and \lndefinition{actions}{define}
  \actionsdef{,}
  where $\S$ and $\A$ are finite.
  \lndefinition{rewards}{Define}
  \rewardsdefg{\lnnote{fnnot}{$\rewardsdefnot$} $\defeq$}{.}

  \lndefinition{steplim}{Define}
  \steplimdef{,}
  \lndefinition{str}{define}
  \strdef{,}
  and \lndefinition{strs}{define}
  \strsdef{.}
  Note that \lnproof{npol}{$|\ST| = \T \times |\S| \times |\A|$}.

  \lndefinition{inddef}{Define} \inddefdefg{\lnnote{paramprob}{$\inddefp$} $\defeq$}{.}
  \lndefinition{pactdef}{Define} \pactdefdef{.}
  \lndefinition{trddef}{Define} \trddefdef{.}

  Now, suppose the \agt\ is following some \str\ $\STT$.
  \lndefinition{rvst}{Define} \rvstdef{,}
  and \lndefinition{rvat}{define} 
  \rvatdef{,}
  whose distributions are dependent on $\STT$.

  Let $\PSTT(\cdot)$ and $\ESTT[\cdot]$ denote the probability of an event
  and the expectation of a quantity
  when the \agt\ follows $\STT$.
  Because we follow a fixed \str, \lndefinition{pact}{we know that}, \pactdeffull{,}
  \lnproof{actcind}{from which it follows} that
  $\AV_\TT$ is independent of all previous \til{}s and \act{}s given $\SV_\TT$.
  Notationally, \actcindpropfullg{\ctext{\lnnote{implgen}{$\actcindprop$}.}}
  \lndefinition{ind}{We know that} \inddef{,}
  and
  \lndefinition{trd}{we know that} \trddef{,}
  \lndefinition{trdcind}{where} \trdcinddef{.}

  We can now write the \trwd\ as
  $$\sum_{\TT = 1}^\T \RW(\SV_\TT)$$
  and the \atrwd\ as
  $$\ESTT\left[\sum_{\TT = 1}^\T \RW(\SV_\TT)\right].$$
\end{part}


\begin{document}
\title{
\hlc[lightlightlightgray]{Reinforcement Learning Theory}\\
\vspace{10pt}
\large 
Rishikesh Vaishnav\\ 
\vspace{10pt}
\normalsize 
Adapted from
{\it Neuro-Dynamic Programming}
by \\
Dmitri P. Bertsekas
and
John N. Tsitsiklis.
}
\maketitle

\begin{part} {Motivating example}
  \def\agent#1{\draw (center) node [rectangle, draw=black, fill=blue, inner sep=0.14cm, #1] {};}

\def\tilecoords#1#2{\draw ($(#1 * \tilesep,#2 * \tilesep)$) node (center) {};}
\def\tile#1{\draw (center) node (#1) [rectangle, draw=black, inner sep=1cm] {};}
\def\tileidx#1{\draw ($(center) - (0.5, 0.5)$) node [rectangle, draw=black, minimum size=0.4cm] {{#1}};}

\def\tileandidx#1{\tile{#1}\tileidx{#1}}

\def\tilecenter#1#2{\draw (#1, #2) node (center) {};}
\def\tiledir#1#2{\draw [->, #1] (center) -- +#2;}
\def\tileup#1{\tiledir{#1}{(0, -0.5)}}
\def\tiledown#1{\tiledir{#1}{(0, 0.5)}}
\def\tileright#1{\tiledir{#1}{(0.5, 0)}}
\def\tileleft#1{\tiledir{#1}{(-0.5, 0)}}

\def\tiletransition#1#2#3#4{\draw [->, opacity=#4] (#1) -- node [#3] {#4} (#2);}

\def\tilesep{2}

\def\actions{
  \tileup{}
  \tiledown{}
  \tileright{}
  \tileleft{}
}

\def\tileloop#1{
  \foreach \row in {0,1,...,2}
  {\sc{row}{\row}
    \foreach \col in {0,1,...,2}
    {\sc{col}{\col}
      \tilecoords{\col}{\row}
      \setidx
      #1
    }
  }
}

\def\board#1{
  \def\setidx{\sc{idx}{(\value{col} + \value{row} * 3) + 1}}
  \tileloop {\tileandidx{\theidx}}
  #1
}

\def\tilerwdraw#1#2{
    \draw #2 node [rectangle, fill=lightlightgray, minimum size=0.4cm] {#1};
}
\def\tilerwd#1{
  \tilerwdraw{#1}{($(center) - (-0.5, 0.5)$)}
}
\def\tileact#1{
  \ifbool{#1 == 0}{ \tileup{strategyact} }{}
  \ifbool{#1 == 1}{ \tileright{strategyact} }{}
  \ifbool{#1 == 2}{ \tiledown{strategyact} }{}
  \ifbool{#1 == 3}{ \tileleft{strategyact} }{}
  \agent{strategyagt}
}
\def\fullboard#1{
  \board
  {
    \input{\curr/_input/rewards.tex}
    \tileloop 
    {
      \tilecoords{\col}{\row}
      \tilerwd{\getl{reward\theidx}};
    }
    #1
  }
}
\def\fullboardact#1{
  \fullboard{
    \input{\curr/_input/actions_#1.tex}
    \tileloop 
    {
      \tilecoords{\col}{\row}
      \tileact{\getl{action\theidx}};
    }
  }
}

\pgfkeys{/tikz/board/.style={y=-1cm, scale=1, every node/.style={transform shape}}}
\pgfkeys{/tikz/strategy/.style={opacity=0.35}}
\pgfkeys{/tikz/strategyagt/.style={strategy}}
\pgfkeys{/tikz/strategyact/.style={strategy, draw opacity=0.7, draw=blue}}

\nc{row}
\nc{col}
\nc{idx}


  Consider the following ``\brd'':\\
  \begin{center}
    \begin{tikzpicture}[board]
      \board{
        \tilecoords{0}{0}\actions\agent{}
      }
    \end{tikzpicture},
  \end{center}
  where there are a total of 9 ``\til{}s'' (\tikz[board, scale=0.1]{\tilecoords{0}{0}\tile{0}\tileidx{$\cdot$}}), 
  an ``\agt'' (\tikz[board, scale=0.7]{\tilecoords{0}{0}\agent{}}), and
  a set of ``\act{}s'' ($\rightarrow$) that the \agt\ can take at any \til.

  At every ``\stp'' in time, the agent is in one of the \til{}s,
  and must take one of the \act{}s. 

  At any \stp, if the \agt\ takes an \act\ in a \til, 
  it will be in a \til\ at the next \stp\ according to a ``\trd''
  (that only depends on the current \til\ and \act, not the current \stp).

  For example, the \agt\ could
  \begin{itemize}
    \def\tilesep{3.2}
    \def\tscale{0.6}
    \item take \act\ $\uparrow$ at \til\ $7$:
      \begin{center}
        \begin{tikzpicture}[board, scale=\tscale]
          \tilecoords{0}{1}
          \tileandidx{7}
          \tileup{draw=blue}
          \agent{}

          \tilecoords{0}{0}
          \tileandidx{4}

          \tilecoords{1}{1}
          \tileandidx{8}

          \tiletransition{7}{8}{below}{0.25}
          \tiletransition{7}{4}{left}{0.75}
        \end{tikzpicture},
      \end{center}
      and
    \item take \act\ $\rightarrow$ at \til\ $5$:
      \begin{center}
        \begin{tikzpicture}[board, scale=\tscale]
          \tilecoords{1}{1}
          \tileandidx{5}
          \tileright{draw=blue}
          \agent{}

          \tilecoords{0}{1}
          \tileandidx{4}

          \tilecoords{2}{1}
          \tileandidx{6}

          \tilecoords{1}{0}
          \tileandidx{2}

          \tilecoords{1}{2}
          \tileandidx{8}

          \tiletransition{5}{6}{below}{0.75}
          \tiletransition{5}{4}{above}{0.05}
          \tiletransition{5}{2}{left}{0.10}
          \tiletransition{5}{8}{right}{0.10}
        \end{tikzpicture},
      \end{center}
  \end{itemize}
  showing the probability of 
  being in another \til\ at 
  the next \stp\ according to the \trd.

  However, this begs the question, 
  what \til\ is the \agt\ at initially (\stp\ 1)? 
  To decide this, there is also an \ind\ over the \til{}s.

  \newpage
  Now, let's make things a bit more interesting. 

  Let every \til\ also have a \rwd\ that 
  the \agt\ recieves from it, 
  leaving our \brd\ looking something like:

  \begin{center}
    \begin{tikzpicture}[board]
      \fullboard{
        \tilecoords{0}{0}\actions\agent{}
      }
    \end{tikzpicture},
  \end{center}
  with the \rwd{}s 
  (\tikz[board, scale=0.5]{\tilecoords{0}{0}\tilerwdraw{$\cdot$}{(0,0)}}) shown.

  Also, let there be some fixed ``\lmt'' such that,
  when the current \stp\ reaches it, everything ends.

  Now, suppose you can ``program'' the \agt\ as follows: 
  for every \til\ and \stp, you can fix the \act\ that the \agt\ takes.
  For example, here's what you might program the \agt\ to do 
  at every \til\ for two different \stp{}s:
  \begin{center}
    \begin{tikzpicture}[board]
      \fullboardact{1}{}
    \end{tikzpicture}
    \hspace{1cm}
    \begin{tikzpicture}[board]
      \fullboardact{2}{.}
    \end{tikzpicture}.
  \end{center}
  We'll call each such programming a ``\str''. 

  We know that the \agt\ starts in some \til\ according to the \ind.
  After this, we follow the \str, 
  transitioning \til{}s according to the \trd, recieving \rwd{}s 
  at every \til\ until
  we reach the \lmt. Let's refer to this process as a ``\run'', and
  the sum of the \rwd{}s over all a \run's \til{}s
  as the ``\trwd''.

  Because there is some \ind\ and \trd\ 
  that are fixed over \run{}s, 
  we know there must be 
  some \atrwd\ for every \str.

  Now, consider the following question: 
  what \str\ has the greatest \atrwd?
\end{part}

\newpage
\begin{part}{Notation}
  To answer this question, let's establish some notation.

  Let's denote
  $$\text{$\S \defeq$ the set of \til{}s},$$
  and 
  $$\text{$\A \defeq$ the set of \act{}s.}$$
  Assume that $\S$ and $\A$ are finite.
  Denote
  $$\text{$\RW : \S \rightarrow \R \defeq$ mapping from \til{}s to \rwd{}s.}$$

  Denote
  $$\text{$\T \defeq$ the \lmt, where $\T \ge 1$},$$
  $$\text{$\STT = \left(\edots\STT{1}{\T}\right) \defeq$ an arbitrary \str, 
  where each mapping $\STT_\TT : \S \rightarrow \A$},$$
  and
  $$\text{$\ST \defeq$ the set of all \strs\ $\pi$}.$$
  Note that \lnproof{num_policies}{$|\ST| = \T \times |\S| \times |\A|$}.

  Now, suppose the \agt\ is following some \str\ $\STT$.
  For all $1 \le \TT \le \T$,
  denote
  $$\text{$\S_\TT \defeq$ random variable for the \til\ at \stp\ $\TT$},$$
  and
  $$\text{$\A_\TT \defeq$ random variable for the \act\ at \stp\ $\TT$},$$
  whose distributions are dependent on $\STT$.

  Let $\PSTT(\cdot)$ and $\ESTT(\cdot)$ denote the probability of an event
  and the expectation of a quantity
  when the \agt\ follows $\STT$.
  For all $\SS \in \S$, we know that
  $$\PSTT(\S_1 = \SS)$$
  is defined by the \ind, and
  for all $\SS, \SS' \in \S$, $2 \le \TT \le \T$ 
  \lnnote{reachability_req}{such that $\PSTT(\S_{\TT - 1} = \SS') > 0$},
  $$\PSTT(\S_\TT = \SS \mid \A_\TTM = \STT_\TTM(\SS'), \S_\TTM = \SS')$$
  is defined by the \trd, where 
  $\S_\TT$ is independent of all previous \til{}s and \act{}s given $\A_\TTM$ and $\S_\TTM$.
  Notationally, for all $2 \le \TT \le \T$,
  %$$ \left( \S_\TT \independent \A_{\TT - 2}, \S_{\TT - 2}, \ldots, \A_{1}, \S_{1} \right)\mid \A_\TTM, \S_\TTM.$$
  $$ 
  \PSTT\left( \S_\TT \mid \A_{\TTM}, \S_{\TTM}, \ldots, \A_{1}, \S_{1} \right)
  = 
  \PSTT\left( \S_\TT \mid \A_{\TTM}, \S_{\TTM}\right)
  $$
  (here we use \lnnote{impl_assgn}{implied assignment}).
  Because we follow a fixed \str, we know that
  $$\PSTT(\A_\TT = \STT_\TT(\SS') \mid \S_\TT = \SS') = 1$$
  for all
  $1 \le \TT \le \T - 1$.

  We can now write the \trwd\ as
  $$\sum_{\TT = 1}^\T \RW(\S_\TT)$$
  and the \atrwd\ as
  $$\ESTT\left[\sum_{\TT = 1}^\T \RW(\S_\TT)\right].$$
\end{part}
\end{document}
